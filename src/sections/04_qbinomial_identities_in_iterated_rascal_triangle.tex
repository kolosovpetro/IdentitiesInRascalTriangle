Consider the table of differences of binomial coefficients and iterated rascal numbers one more time
as there is another pattern we can spot.
\begin{figure}[H]
    \centering
    \includegraphics[width=1\textwidth]{../img/03_Difference_Binomial_Rascal_i_3_OneQBinomialCoefficients}
    ~\caption{Difference $\binom{n}{k}-\rascalNumber{n}{k}{3}$.
    Highlighted column is $(1,5)$-binomial coefficient $\oneQBinomial{n}{k}{5}$.
    Sequence \href{https://oeis.org/A096943}{\texttt{A096943}} in the OEIS~\cite{sloane2004sixth}.}
    \label{fig:difference-qbinomial-rascal-i-3}
\end{figure}
The $(1,q)$-binomial coefficients $\oneQBinomial{n}{k}{q}$ are special kind of binomial coefficients defined by
\begin{definition}
    $(1,q)$-Binomial coefficient~\cite{sloane2004pascal}
    \begin{equation}
        \oneQBinomial{n}{k}{q} =
        \begin{cases}
            q & \mathrm{if} \; k=0, n=0 \\
            1 & \mathrm{if} \; k=0 \\
            0 & \mathrm{if} \; k > n \\
            \oneQBinomial{n-1}{k}{q} + \oneQBinomial{n-1}{k-1}{q}
        \end{cases}\label{eq:qbinomial-definition}
    \end{equation}
\end{definition}
Indeed, the relation shown in Figure~\eqref{fig:difference-qbinomial-rascal-i-3} is true for every $i$,
so that it establishes a relation between $(1,q)$-binomial coefficients and iterated rascal numbers.
\begin{proposition} (Relation between iterated rascal numbers and $(1,q)$-binomial coefficients)
    For every fixed $i\geq0$
    \label{prop:row-column-difference-qbinomial}
    \begin{align*}
        \binom{2i+3+j}{i+2} - \rascalNumber{2i+3+j}{i+2}{i} &= \oneQBinomial{i+2+j}{i+2}{i+2}
    \end{align*}
\end{proposition}
