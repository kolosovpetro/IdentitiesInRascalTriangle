In 2010, three middle school students, Alif Anggoro, Eddy Liu, and Angus Tulloch~\cite{anggoro2010rascal},
were challenged to provide the next row for the number triangle shown below:
\[
    \begin{array}{cccccccc}
        &   &   &   & 1 &   &   &   \\
        &   &   & 1 &   & 1 &   &   \\
        &   & 1 &   & 2 &   & 1 &   \\
        & 1 &   & 3 &   & 3 &   & 1 \\
    \end{array}
\]

While the expected answer was ``1 4 6 4 1'' Anggoro, Liu, and Tulloch suggested ``1 4 5 4 1'' instead.
They devised this new row via so-called diamond formula:
\begin{align*}
    \south  = \frac{\east \cdot \west + 1}{\north}
\end{align*}

So that upcoming rows of the triangle are
\begin{table}[H]
    \begin{center}
        \setlength\extrarowheight{-6pt}
        \begin{tabular}{c|cccccccc}
            $n/k$ & 0 & 1 & 2  & 3  & 4  & 5  & 6 & 7 \\
            \hline
            0     & 1 &   &    &    &    &    &   &   \\
            1     & 1 & 1 &    &    &    &    &   &   \\
            2     & 1 & 2 & 1  &    &    &    &   &   \\
            3     & 1 & 3 & 3  & 1  &    &    &   &   \\
            4     & 1 & 4 & 5  & 4  & 1  &    &   &   \\
            5     & 1 & 5 & 7  & 7  & 5  & 1  &   &   \\
            6     & 1 & 6 & 9  & 10 & 9  & 6  & 1 &   \\
            7     & 1 & 7 & 11 & 13 & 13 & 11 & 7 & 1
        \end{tabular}
    \end{center}
    \caption{Rascal triangle. See the OEIS sequence \href{https://oeis.org/A077028}{\texttt{A077028}}~\cite{sloane2002rascal}.}
    \label{tab:rascal-triangle-i-1}
\end{table}


Since then, a lot of work has been done over the topic of rascal triangles.
Numerous identities and relations have been revealed.
For instance, few combinatorial interpretations of rascal numbers provided at~\cite{gibbs2024two}, in particular,
these interpretations establish a relation between rascal numbers and combinatorics of binary words.
Few generalization approaches were proposed, namely generalized
and iterated rascal triangles~\cite{hotchkiss2019generalized,gregory2023iterated}.
