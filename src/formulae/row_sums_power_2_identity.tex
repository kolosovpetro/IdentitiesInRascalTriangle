By definition
\begin{align}
    \rascalNumber{n}{k}{i} &= \sum_{m=0}^{i} \binom{n-k}{m} \binom{k}{m}
\end{align}
Consider Vandermonde convolution
\begin{equation*}
    \binom{a+b}{r} = \sum_{m=0}^{r} \binom{a}{m} \binom{b}{r-m}
\end{equation*}
Then generalized rascal number is partial case of Vandermonde convolution with upper summation bound equals $i$
\begin{equation*}
    \rascalNumber{n}{k}{i} = \sum_{m=0}^{i} \binom{n-k}{m} \binom{k}{m} = \sum_{m=0}^{i} \binom{n-k}{m} \binom{k}{k-m}
\end{equation*}
Difference between binomial coefficients and iterated rascal numbers
\begin{align*}
    \binom{n}{k} - \rascalNumber{n}{k}{i} &= \sum_{m=i+1}^{k} \binom{n-k}{m} \binom{k}{k-m}
\end{align*}
\begin{proposition}
    Row $4i+3$ sum gives $2^{4i+2}$
    \begin{align*}
        \sum_{k=0}^{4i+3} \rascalNumber{4i+3}{k}{i} &= 2^{4i+2}
%        \sum_{k=0}^{4i+3} \rascalNumber{4i+3}{k}{i} &= \sum_{k=0}^{4i+3} \sum_{m=0}^{i} \binom{4i+3-k}{m} \binom{k}{m} = 2^{4i+2}
    \end{align*}
\end{proposition}
\begin{align*}
    \sum_{k=0}^{4i+3} \rascalNumber{4i+3}{k}{i} &= 2^{4i+2} \\
    \sum_{k=0}^{4i+3} \rascalNumber{4i+3}{k}{i} &= \sum_{k=0}^{4i+3} \sum_{m=0}^{i} \binom{4i+3-k}{m} \binom{k}{m} = 2^{4i+2}
\end{align*}
We know that sum of binomial coefficients equals $2^n$
\begin{equation*}
    \sum_{k=0}^{4i+3} \binom{4i+3}{k}  = 2^{4i+3}
\end{equation*}
If conjecture holds, then it must be true that
\begin{equation*}
    \sum_{k=0}^{4i+3} \binom{4i+3}{k} - \rascalNumber{4i+3}{k}{i} = 2^{4i+2}
\end{equation*}
because $2^n - 2^n = 2^{n-1}$.
Thus,
\begin{equation*}
    \sum_{k=0}^{4i+3} \sum_{m=i+1}^{k} \binom{4i+3-k}{m} \binom{k}{m} = 2^{4i+2}
\end{equation*}
In case $k<i$ the sum $\sum_{m=i+1}^{k} \binom{4i+3-k}{m}$ is always zero.
\begin{equation*}
    \sum_{k=i+1}^{4i+3} \sum_{m=i+1}^{k} \binom{4i+3-k}{m} \binom{k}{m} = 2^{4i+2}
\end{equation*}
\begin{equation*}
    \sum_{k=0}^{4i+3} \sum_{m=i+1}^{k} \binom{4i+3-(i+1+k)}{m} \binom{k}{m} = 2^{4i+2}
\end{equation*}
\begin{equation*}
    \sum_{k=0}^{4i+3} \sum_{m=i+1}^{k} \binom{3i+2-k}{m} \binom{k}{m} = 2^{4i+2}
\end{equation*}
\begin{equation*}
    \sum_{k=0}^{4i+3} \sum_{m=0}^{k} \binom{3i+2-k}{i+1+m} \binom{k}{i+1+m} = 2^{4i+2}
\end{equation*}
