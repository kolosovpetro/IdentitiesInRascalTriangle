%! suppress = MissingLabel
\documentclass[12pt,letterpaper,oneside,reqno]{amsart}
\usepackage{amsfonts}
\usepackage{amsmath}
\usepackage{amssymb}
\usepackage{amsthm}
\usepackage{float}
\usepackage{mathrsfs}
\usepackage{colonequals}
\usepackage[font=small,labelfont=bf]{caption}
\usepackage[left=1in,right=1in,bottom=1in,top=1in]{geometry}
\usepackage[pdfpagelabels,hyperindex,colorlinks=true,linkcolor=blue,urlcolor=magenta,citecolor=green]{hyperref}
\usepackage{graphicx}
\linespread{1.7}
\emergencystretch=1em
\usepackage{array}
\usepackage{etoolbox}
\apptocmd{\sloppy}{\hbadness 10000\relax}{}{}
\raggedbottom

\newcommand \anglePower [2]{\langle #1 \rangle \sp{#2}}
\newcommand \bernoulli [2][B] {{#1}\sb{#2}}
\newcommand \curvePower [2]{\{#1\}\sp{#2}}
\newcommand \coeffA [3][A] {{\mathbf{#1}} \sb{#2,#3}}
\newcommand \polynomialP [4][P]{{\mathbf{#1}}\sp{#2} \sb{#3}(#4)}

% ordinary derivatives
\newcommand \derivative [2] {\frac{d}{d #2} #1}                              % 1 - function; 2 - variable;
\newcommand \pderivative [2] {\frac{\partial #1}{\partial #2}}               % 1 - function; 2 - variable;
\newcommand \qderivative [1] {D_{q} #1}                                      % 1 - function
\newcommand \nqderivative [1] {D_{n,q} #1}                                   % 1 - function
\newcommand \qpowerDerivative [1] {\mathcal{D}_q #1}                         % 1 - function;
\newcommand \finiteDifference [1] {\Delta #1}                                % 1 - function;
\newcommand \pTsDerivative [2] {\frac{\partial #1}{\Delta #2}}               % 1 - function; 2 - variable;

% high order derivatives
\newcommand \derivativeHO [3] {\frac{d^{#3}}{d {#2}^{#3}} #1}                % 1 - function; 2 - variable; 3 - order
\newcommand \pderivativeHO [3]{\frac{\partial^{#3}}{\partial {#2}^{#3}} #1}
\newcommand \qderivativeHO [2] {D_{q}^{#2} #1}                               % 1 - function; 2 - order
\newcommand \qpowerDerivativeHO [2] {\mathcal{D}_{q}^{#2} #1}                % 1 - function; 2 - order
\newcommand \finiteDifferenceHO [2] {\Delta^{#2} #1}                         % 1 - function; 2 - order
\newcommand \pTsDerivativeHO [3] {\frac{\partial^{#3}}{\Delta {#2}^{#3}} #1} % 1 - function; 2 - variable;

% central factorials and related symbols
\newcommand \centralFactorial [2] {#1^{[#2]}}
\newcommand \fallingFactorial [2] {\left(#1 \right)^{\underline{#2}}}
\newcommand{\stirlingii}{\genfrac{\{}{\}}{0pt}{}}
\newcommand{\eulerianNumber}{\genfrac{\langle}{\rangle}{0pt}{}}

% rascal numbers etc
\newcommand \rascalNumber [3] {\binom{#1}{#2}_{#3}}
\newcommand \north[0] {\mathbf{North}}
\newcommand \south[0] {\mathbf{South}}
\newcommand \west[0] {\mathbf{West}}
\newcommand \east[0] {\mathbf{East}}

% 1-q pascal notation


\newcommand{\genstirlingI}[3]{%
    \genfrac{[}{]}{0pt}{#1}{#2}{#3}%
}
\newcommand{\genstirlingII}[3]{%
    \genfrac{\{}{\}}{0pt}{#1}{#2}{#3}%
}
\newcommand{\oneQBinomial}[3]{\genstirlingI{}{#1}{#2}^{#3}}


% for llceil coeffcient
\newcommand{\nobarfrac}{\genfrac{}{}{0pt}{}}
\def\llceil{\left\lceil\kern-3.5pt\left\lceil}
\def\rrfloor{\right\rfloor\kern-3.5pt\right\rfloor}
\newcommand \llceilCoefficient [3] {\llceil \nobarfrac{#1}{#2} \rrfloor_{#3}}


\newtheorem{thm}{Theorem}[section]
\newtheorem{cor}[thm]{Corollary}
\newtheorem{lem}[thm]{Lemma}
\newtheorem{examp}[thm]{Example}
\newtheorem{conj}[thm]{Conjecture}
\newtheorem{defn}[thm]{Definition}

\numberwithin{equation}{section}

\title[Polynomial identities involving Rascal Triangle]
{Polynomial identities involving Rascal Triangle}
\author[Petro Kolosov]{Petro Kolosov}
\address{Software Developer, DevOps Engineer}
\email{kolosovp94@gmail.com}
\urladdr{https://kolosovpetro.github.io}
\keywords{
    Keyword1, Keyword2
}
\subjclass[2010]{26E70, 05A30}
\date{\today}
\hypersetup{
    pdftitle={LaTeX Template for Github},
    pdfsubject={
        Your Subject List
    },
    pdfauthor={Petro Kolosov},
    pdfkeywords={
        Your Keywords list
    }
}
\begin{document}
\begin{abstract}
    Abstract
\end{abstract}

\maketitle


\section{Definitions}
Definition of generalized Rascal triangle
\begin{equation}
    \rascalNumber{n}{k}{i} = \sum_{m=0}^{i} \binom{n-k}{m} \binom{k}{m}
\end{equation}
Definition of $(1, q)$-Pascal triangle
\begin{equation*}
    \oneQBinomial{n}{k}{q} =
    \begin{cases}
        q & \text{if } k=0, n=0 \\
        1 & \text{if } k=0 \\
        0 & \text{if } k > n \\
        \oneQBinomial{n-1}{k}{q} + \oneQBinomial{n-1}{k-1}{q}
    \end{cases}
\end{equation*}


\section{Sides of world}
\begin{align*}
    \north &= \rascalNumber{n-2}{k-1}{i} \\
    \south &= \rascalNumber{n}{k}{i} \\
    \west  &= \rascalNumber{n-1}{k-1}{i} \\
    \east  &= \rascalNumber{n-1}{k}{i}
\end{align*}
Identity see Hotchkiss
\begin{align*}
    \south                  &= \frac{\east \cdot \west + 1}{\north} \\
    \rascalNumber{n}{k}{i}  &= \frac{\rascalNumber{n-1}{k}{i} \rascalNumber{n-1}{k-1}{i} +1}{\rascalNumber{n-2}{k-1}{i}}
\end{align*}
Identity see Hotchkiss, for all inner $k > 0$ and $k < n$
\begin{align*}
    \south                 &= \east + \west - \north  + 1 \\
    \rascalNumber{n}{k}{i} &= \rascalNumber{n-1}{k}{i} + \rascalNumber{n-1}{k-1}{i} - \rascalNumber{n-2}{k-1}{i} + 1
\end{align*}


\section{Formulae}\label{sec:formulae}
Claim 1
\begin{equation}
    \rascalNumber{n}{k}{i} = \binom{n}{k}, \quad 0 \leq k \leq i
\end{equation}
Claim 2
\begin{equation}
    \rascalNumber{n}{k}{i} = \binom{n}{k}, \quad 0 \leq n \leq 2i+1
\end{equation}
Claim 3
\begin{equation}
    \binom{j}{k} - \rascalNumber{j}{k}{i} = \binom{n}{i+1}, \quad j \geq 2i+2, k=i+1
\end{equation}

\begin{equation}
    \binom{2i+j+2}{i+1} - \rascalNumber{2i+j+2}{i+1}{i} = \binom{i+j+1}{i+1}
\end{equation}

\begin{equation}
    \binom{2(i+1) +j}{i+1} - \rascalNumber{2(i+1) +j}{i+1}{i} = \binom{(i+1)+j}{i+1}
\end{equation}

\begin{equation}
    \binom{2t +j}{t} - \rascalNumber{2t +j}{t}{t-1} = \binom{t+j}{t}
\end{equation}



\end{document}
