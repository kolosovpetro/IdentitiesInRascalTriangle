%! suppress = MissingLabel
\documentclass[12pt,letterpaper,oneside,reqno]{amsart}
\usepackage{amsfonts}
\usepackage{amsmath}
\usepackage{amssymb}
\usepackage{amsthm}
\usepackage{float}
\usepackage{mathrsfs}
\usepackage{colonequals}
\usepackage[font=small,labelfont=bf]{caption}
\usepackage[left=1in,right=1in,bottom=1in,top=1in]{geometry}
\usepackage[pdfpagelabels,hyperindex,colorlinks=true,linkcolor=blue,urlcolor=magenta,citecolor=green]{hyperref}
\usepackage{graphicx}
\linespread{1.7}
\emergencystretch=1em
\usepackage{array}
\usepackage{etoolbox}
\apptocmd{\sloppy}{\hbadness 10000\relax}{}{}
\raggedbottom

\newcommand \anglePower [2]{\langle #1 \rangle \sp{#2}}
\newcommand \bernoulli [2][B] {{#1}\sb{#2}}
\newcommand \curvePower [2]{\{#1\}\sp{#2}}
\newcommand \coeffA [3][A] {{\mathbf{#1}} \sb{#2,#3}}
\newcommand \polynomialP [4][P]{{\mathbf{#1}}\sp{#2} \sb{#3}(#4)}

% ordinary derivatives
\newcommand \derivative [2] {\frac{d}{d #2} #1}                              % 1 - function; 2 - variable;
\newcommand \pderivative [2] {\frac{\partial #1}{\partial #2}}               % 1 - function; 2 - variable;
\newcommand \qderivative [1] {D_{q} #1}                                      % 1 - function
\newcommand \nqderivative [1] {D_{n,q} #1}                                   % 1 - function
\newcommand \qpowerDerivative [1] {\mathcal{D}_q #1}                         % 1 - function;
\newcommand \finiteDifference [1] {\Delta #1}                                % 1 - function;
\newcommand \pTsDerivative [2] {\frac{\partial #1}{\Delta #2}}               % 1 - function; 2 - variable;

% high order derivatives
\newcommand \derivativeHO [3] {\frac{d^{#3}}{d {#2}^{#3}} #1}                % 1 - function; 2 - variable; 3 - order
\newcommand \pderivativeHO [3]{\frac{\partial^{#3}}{\partial {#2}^{#3}} #1}
\newcommand \qderivativeHO [2] {D_{q}^{#2} #1}                               % 1 - function; 2 - order
\newcommand \qpowerDerivativeHO [2] {\mathcal{D}_{q}^{#2} #1}                % 1 - function; 2 - order
\newcommand \finiteDifferenceHO [2] {\Delta^{#2} #1}                         % 1 - function; 2 - order
\newcommand \pTsDerivativeHO [3] {\frac{\partial^{#3}}{\Delta {#2}^{#3}} #1} % 1 - function; 2 - variable;

% central factorials and related symbols
\newcommand \centralFactorial [2] {#1^{[#2]}}
\newcommand \fallingFactorial [2] {\left(#1 \right)^{\underline{#2}}}
\newcommand{\stirlingii}{\genfrac{\{}{\}}{0pt}{}}
\newcommand{\eulerianNumber}{\genfrac{\langle}{\rangle}{0pt}{}}

% rascal numbers etc
\newcommand \rascalNumber [3] {\binom{#1}{#2}_{#3}}
\newcommand \north[0] {\mathbf{North}}
\newcommand \south[0] {\mathbf{South}}
\newcommand \west[0] {\mathbf{West}}
\newcommand \east[0] {\mathbf{East}}

% 1-q pascal notation


\newcommand{\genstirlingI}[3]{%
    \genfrac{[}{]}{0pt}{#1}{#2}{#3}%
}
\newcommand{\genstirlingII}[3]{%
    \genfrac{\{}{\}}{0pt}{#1}{#2}{#3}%
}
\newcommand{\oneQBinomial}[3]{\genstirlingI{}{#1}{#2}^{#3}}


% for llceil coeffcient
\newcommand{\nobarfrac}{\genfrac{}{}{0pt}{}}
\def\llceil{\left\lceil\kern-3.5pt\left\lceil}
\def\rrfloor{\right\rfloor\kern-3.5pt\right\rfloor}
\newcommand \llceilCoefficient [3] {\llceil \nobarfrac{#1}{#2} \rrfloor_{#3}}


\newtheorem{thm}{Theorem}[section]
\newtheorem{cor}[thm]{Corollary}
\newtheorem{lem}[thm]{Lemma}
\newtheorem{examp}[thm]{Example}
\newtheorem{conj}[thm]{Conjecture}
\newtheorem{defn}[thm]{Definition}

\numberwithin{equation}{section}

\title[Polynomial identities involving Rascal Triangle]
{Polynomial identities involving Rascal Triangle}
\author[Petro Kolosov]{Petro Kolosov}
\address{Software Developer, DevOps Engineer}
\email{kolosovp94@gmail.com}
\urladdr{https://kolosovpetro.github.io}
\keywords{
    Keyword1, Keyword2
}
\subjclass[2010]{26E70, 05A30}
\date{\today}
\hypersetup{
    pdftitle={LaTeX Template for Github},
    pdfsubject={
        Your Subject List
    },
    pdfauthor={Petro Kolosov},
    pdfkeywords={
        Your Keywords list
    }
}
\begin{document}
\begin{abstract}
    Abstract
\end{abstract}

\tableofcontents

\maketitle


\section{Definitions}
Definition of generalized Rascal triangle
\begin{align}
    \rascalNumber{n}{k}{i} &= \sum_{m=0}^{i} \binom{n-k}{m} \binom{k}{m} = \sum_{m=0}^{i} \binom{n-k}{m} \binom{k}{k-m} \\
    &= \binom{n-k}{0} \binom{k}{0} + \binom{n-k}{1} \binom{k}{1} + \binom{n-k}{2} \binom{k}{2} + \ldots + \binom{n-k}{i} \binom{k}{i}
\end{align}
Definition of $(1, q)$-Pascal triangle
\begin{equation*}
    \oneQBinomial{n}{k}{q} =
    \begin{cases}
        q & \text{if } k=0, n=0 \\
        1 & \text{if } k=0 \\
        0 & \text{if } k > n \\
        \oneQBinomial{n-1}{k}{q} + \oneQBinomial{n-1}{k-1}{q}
    \end{cases}
\end{equation*}
Pascals triangle as polynomial
\begin{equation}
    \binom{n}{k}= \frac{\fallingFactorial{n}{k}}{k!}= \frac{1}{k!} n(n-1)(n-2)\cdots (n-(k-1)) = \prod_{i=1}^{k} \frac{n-i+1}{i}
\end{equation}

\section{Formulae}\label{sec:formulae}

\subsection{Claim 0. Vandermonde convolution} Generalized rascal triangle is partial case of Chu-Vandermonde convolution

\begin{equation*}
    \binom{a+b}{r} = \sum_{m=0}^{r} \binom{a}{m} \binom{b}{r-m}
\end{equation*}

\subsection{Claim 1. I-th column identity} Generalized rascal triangle equals to Pascal's triangle up to $i$-th column
\begin{align}
    \rascalNumber{n}{k}{i}          &= \binom{n}{k}, \quad 0 \leq k \leq i \\
    \rascalNumber{n}{i-j}{i}        &= \binom{n}{i-j}, \quad ColumnIdentity1 \\
    \rascalNumber{n}{n-i+j}{i}      &= \binom{n}{n-i+j}, \quad ColumnIdentity2 \\
\end{align}

\subsection{Claim 2. 2i+1 row identity} Generalized rascal triangle equals to Pascal's triangle up to $2i+1$-th row
\begin{align}
    \rascalNumber{n}{k}{i}                  &= \binom{n}{k}, \quad 0 \leq n \leq 2i+1
\end{align}
For every fixed $i \geq 0$
\begin{align}
    \rascalNumber{2i+1-j}{k}{i}             &= \binom{2i+1-j}{k} \quad RowIdentity1 \\
    \rascalNumber{(2i+1)-j}{k}{(2i+1)-i-1}  &= \binom{(2i+1)-j}{k}
\end{align}
For every fixed $i \geq 0$ and $t \geq 2i+1$
\begin{align}
    \rascalNumber{t-j}{k}{t-i-1}            = \binom{t-j}{k} \quad RowIdentity2
\end{align}
For $k=j$
\begin{align*}
    \rascalNumber{2i+1-j}{j}{i}         &= \binom{2i+1-j}{j}, \quad 0 \leq j \leq i \quad RowIdentity3 \\
    \rascalNumber{2i+1-j}{2i+1-2j}{i}   &= \binom{2i+1-j}{2i+1-2j} \\
    \rascalNumber{(2i+1)-j}{(2i+1)-2j}{(2i+1)-i-1}   &= \binom{(2i+1)-j}{(2i+1)-2j} \\
    \rascalNumber{t-j}{t-2j}{t-i-1}     &= \binom{t-j}{t-2j}, \quad t \geq 2i+1, \quad 0 \leq j \leq t-i-1, \quad RowIdentity4 \\
\end{align*}

\subsection{Proof of 2i+1 row identity}
Let be definition of rascal number
\begin{equation*}
    \rascalNumber{n}{k}{i} = \sum_{m=0}^{i} \binom{n-k}{m} \binom{k}{m} = \sum_{m=0}^{i} \binom{n-k}{m} \binom{k}{k-m}
\end{equation*}
Then for every $0 \leq n \leq 2i+1$ holds
\begin{align}
    \rascalNumber{n}{k}{i}                  &= \binom{n}{k}, \quad 0 \leq n \leq 2i+1
\end{align}
Let be Vandermonde formula
\begin{equation*}
    \binom{a+b}{r} = \sum_{m=0}^{r} \binom{a}{m} \binom{b}{r-m}
\end{equation*}
By Vandermonde formula
\begin{equation*}
    \binom{n}{k} = \sum_{m=0}^{k} \binom{n-k}{m} \binom{k}{k-m}
\end{equation*}
Then for every $0 \leq n \leq 2i+1$ and $0 \leq k \leq 2i+1-n$
\begin{align*}
    \rascalNumber{2i+1-n}{k}{i} = \binom{2i+1-n}{k}
%    \rascalNumber{2i+1-j}{k}{i} &= \sum_{m=0}^{i} \binom{2i+1-j-k}{m} \binom{k}{m} = \binom{2i+1-j}{k} \\
%    \binom{2i+1-j}{k}           &= \sum_{m=0}^{i} \binom{2i+1-j-k}{m} \binom{k}{m}  \\
%    \binom{2i}{k}               &= \sum_{m=0}^{i} \binom{2i-k}{m} \binom{k}{m}  \\
\end{align*}
Thus we have to prove that
\begin{align*}
%    \rascalNumber{2i+1-n}{k}{i} &= \binom{2i+1-n}{k} \\
    \rascalNumber{2i+1-n}{k}{i} &= \sum_{m=0}^{i} \binom{2i+1-n-k}{m} \binom{k}{m} = \binom{2i+1-n}{k} \\
    \binom{2i+1-n}{k}           &= \sum_{m=0}^{i} \binom{2i+1-n-k}{m} \binom{k}{m} = \sum_{m=0}^{k} \binom{2i+1-n-k}{m} \binom{k}{k-m}
%    \sum_{m=0}^{i} \binom{2i+1-n-k}{m} \binom{k}{m} &= \binom{2i+1-n-k}{0} \binom{k}{0} + \binom{2i+1-n-k}{1} \binom{k}{1} + \binom{2i+1-n-k}{2} \binom{k}{2} \\
%    \binom{2i}{k}               &= \sum_{m=0}^{i} \binom{2i-k}{m} \binom{k}{m}  \\
\end{align*}
Rewrite explicitly gives for $i$
\begin{align*}
    \sum_{m=0}^{i} \binom{2i+1-n-k}{m} \binom{k}{m} &= \binom{2i+1-n-k}{0} \binom{k}{0} + \binom{2i+1-n-k}{1} \binom{k}{1} \\
                                                    &+ \binom{2i+1-n-k}{2} \binom{k}{2} + \binom{2i+1-n-k}{3} \binom{k}{3} \\
                                                    &+ \cdots + \binom{2i+1-n-k}{i} \binom{k}{i}
\end{align*}
Rewrite explicitly gives for $k$
\begin{align*}
    \sum_{m=0}^{k} \binom{2i+1-n-k}{m} \binom{k}{m} &= \binom{2i+1-n-k}{0} \binom{k}{0} + \binom{2i+1-n-k}{1} \binom{k}{1} \\
    &+ \binom{2i+1-n-k}{2} \binom{k}{2} + \binom{2i+1-n-k}{3} \binom{k}{3} \\
    &+ \cdots + \binom{2i+1-n-k}{k} \binom{k}{k}
\end{align*}
We have three cases in general $k < i$, $k=i$, $k > i$, so we have to prove that
\begin{equation*}
    \sum_{m=0}^{k} \binom{2i+1-n-k}{m} \binom{k}{m} - \sum_{m=0}^{i} \binom{2i+1-n-k}{m} \binom{k}{m} = 0
\end{equation*}
For the case $k < i$ proof in Jenna et all paper.
For the case $k=i$ proof is trivial.
Thus,
\begin{equation*}
    \sum_{m=i+1}^{k} \binom{2i+1-n-k}{m} \binom{k}{m} = 0
\end{equation*}
Considering the constraints,
\begin{equation*}
    \begin{cases}
        n \geq 0 \\
        k \geq i+1 \\
        2i+1-n-k \leq i-n \\
        m \geq i+1
    \end{cases}
\end{equation*}
Thus,
\begin{equation*}
    \sum_{m=i+1}^{k} \binom{2i+1-n-k}{m} \binom{k}{m} = 0
\end{equation*}
because binomial coefficients $\binom{i-n-s}{i+1+s}$ are zero for each $i, n, s \geq 0$.
\subsection{Claim 3} Row-column difference identity.
Proof via Vandermonde's identity.
For every fixed $i \geq 1$
\begin{align*}
    \binom{n+2i}{i} - \rascalNumber{n+2i}{i}{i-1} &= \binom{n+i}{i} \quad RowColumnDifferenceIdentity1 \\
    \binom{n+2i}{n+i} - \rascalNumber{n+2i}{n+i}{i-1} &= \binom{n+i}{n} \\
    \binom{(n+i)+i}{(n+i)} - \rascalNumber{(n+i)+i}{(n+i)}{i-1} &= \binom{(n+i)}{(n+i)-i} \\
    \binom{j+i}{j} - \rascalNumber{j+i}{j}{i-1} &= \binom{j}{j-i}, \quad RowColumnDifferenceIdentity2 \\
\end{align*}

\subsection{Claim 4} Relation between $(1,q)$-Pascal's triangle
\begin{align*}
    \binom{2i+3+j}{i+2} - \rascalNumber{2i+3+j}{i+2}{i} &= \oneQBinomial{i+2+j}{i+2}{i+2}, \quad OneQPascalIdentity1 \\
    \binom{2(i+2)-1+j}{i+2} - \rascalNumber{2(i+2)-1+j}{i+2}{i} &= \oneQBinomial{i+2+j}{i+2}{i+2} \\
    \binom{2t-1+j}{t} - \rascalNumber{2t-1+j}{t}{t-2} &= \oneQBinomial{t+j}{t}{t}, \quad OneQPascalIdentity2 \\
\end{align*}

\section{Sides of world}
\begin{align*}
    \north &= \rascalNumber{n-2}{k-1}{i} \\
    \south &= \rascalNumber{n}{k}{i} \\
    \west  &= \rascalNumber{n-1}{k-1}{i} \\
    \east  &= \rascalNumber{n-1}{k}{i}
\end{align*}
Identity see Hotchkiss
\begin{align}
    \south                  &= \frac{\east \cdot \west + 1}{\north} \\
    \rascalNumber{n}{k}{i}  &= \frac{\rascalNumber{n-1}{k}{i} \rascalNumber{n-1}{k-1}{i} +1}{\rascalNumber{n-2}{k-1}{i}}
\end{align}
Identity see Hotchkiss, for all inner $k > 0$ and $k < n$
\begin{align}
    \south                 &= \east + \west - \north  + 1 \\
    \rascalNumber{n}{k}{i} &= \rascalNumber{n-1}{k}{i} + \rascalNumber{n-1}{k-1}{i} - \rascalNumber{n-2}{k-1}{i} + 1
\end{align}



\end{document}
