Rascal triangle is Pascal-like numeric triangle developed in 2010 by three middle school students,
Alif Anggoro, Eddy Liu, and Angus Tulloch~\cite{anggoro2010rascal}.
During math classes they were challenged to provide the next row for the following number triangle
\[
    \begin{array}{cccccccc}
        &   &   &   & 1 &   &   &   \\
        &   &   & 1 &   & 1 &   &   \\
        &   & 1 &   & 2 &   & 1 &   \\
        & 1 &   & 3 &   & 3 &   & 1 \\
        & & & & \dots & &
    \end{array}
\]

The teacher anticipated that the next row would match Pascal's triangle, such as ``1 4 6 4 1'',
by applying the binomial coefficient recurrence rule $\south = \east + \west$.
However, Anggoro, Liu, and Tulloch proposed that the next row should be ``1 4 5 4 1''.
Instead of using Pascal's triangle rule $\south = \east + \west$, they derived this new row using
a relation they termed the diamond formula
\begin{align}
    \south = \frac{\east \cdot \west + 1}{\north}
    \label{eq:diamond-rule}
\end{align}
By applying the recurrence relation from equation~\eqref{eq:diamond-rule},
the students successfully generated an entirely new triangular sequence,
now referred to as the Rascal triangle.
\begin{table}[H]
    \begin{center}
        \begin{tabular}{c|cccccccccc}
            $n/k$ & 0 & 1 & 2  & 3  & 4  & 5  & 6  & 7  & 8 & 9 \\
            \hline
            0     & 1 &   &    &    &    &    &    &    &   &   \\
            1     & 1 & 1 &    &    &    &    &    &    &   &   \\
            2     & 1 & 2 & 1  &    &    &    &    &    &   &   \\
            3     & 1 & 3 & 3  & 1  &    &    &    &    &   &   \\
            4     & 1 & 4 & 5  & 4  & 1  &    &    &    &   &   \\
            5     & 1 & 5 & 7  & 7  & 5  & 1  &    &    &   &   \\
            6     & 1 & 6 & 9  & 10 & 9  & 6  & 1  &    &   &   \\
            7     & 1 & 7 & 11 & 13 & 13 & 11 & 7  & 1  &   &   \\
            8     & 1 & 8 & 13 & 16 & 17 & 16 & 13 & 8  & 1 &   \\
            9     & 1 & 9 & 15 & 19 & 21 & 21 & 19 & 15 & 9 & 1
        \end{tabular}
    \end{center}
    \caption{Rascal triangle. Sequence \href{https://oeis.org/A077028}{\texttt{A077028}} in OEIS~\cite{sloane2002rascal}.}
    \label{tab:rascal-triangle-i-1}
\end{table}

For example, the fourth row is ``1 4 5 4 1'' because $4 = \frac{1 \cdot 3 + 1}{1}$ and $5 = \frac{3 \cdot 3 + 1}{2}$.
Moreover, the Rascal triangle, as presented in table~\eqref{tab:rascal-triangle-i-1},
represents the first and foundational instance of a new family of Pascal-like triangles.
This family, known as \textit{iterated Rascal triangles}, was first introduced by J. Gregory in her
master's thesis~\cite{gregory2022iterated_thesis}.

We define the $k$-th element in the $n$-th row of an iterated Rascal triangle as $\rascalNumber{n}{k}{i}$,
where $i$ represents the number of iterations.
The integer sequence produced by $\rascalNumber{n}{k}{i}$ is referred to as an \textit{iterated Rascal triangle $Ri$},
and each $\rascalNumber{n}{k}{i}$ is termed an \textit{iterated Rascal number}.
Therefore, the Rascal triangle shown in table~\eqref{tab:rascal-triangle-i-1} corresponds
to the iterated Rascal triangle $R1$, generated by the formula $\rascalNumber{n}{k}{1} = k(n-k)+1$.
While the iterated Rascal number $\rascalNumber{n}{k}{i}$ is defined by the diamond rule~\eqref{eq:diamond-rule},
which differs from the standard binomial coefficient recurrence,
it still maintains a significant connection with the binomial coefficients $\binom{n}{k}$,
as demonstrated by
\begin{equation}
    \rascalNumber{n}{k}{i} = \sum_{m=0}^{i} \binom{n-k}{m} \binom{k}{m}
    \label{eq:iterated-rascal-number}
\end{equation}
For example, $\rascalNumber{7}{4}{3}=35$, $\rascalNumber{12}{7}{5}=792$, $\rascalNumber{11}{5}{5}=462$.
\begin{examp}
    \emph{
        Rascal triangle R2 generated by $\rascalNumber{n}{k}{2}$
        \begin{table}[H]
    \begin{center}
        \begin{tabular}{c|cccccccccc}
            $n/k$ & 0 & 1 & 2  & 3  & 4  & 5  & 6  & 7  & 8 & 9 \\
            \hline
            0     & 1 &   &    &    &    &    &    &    &   &   \\
            1     & 1 & 1 &    &    &    &    &    &    &   &   \\
            2     & 1 & 2 & 1  &    &    &    &    &    &   &   \\
            3     & 1 & 3 & 3  & 1  &    &    &    &    &   &   \\
            4     & 1 & 4 & 6  & 4  & 1  &    &    &    &   &   \\
            5     & 1 & 5 & 10 & 10 & 5  & 1  &    &    &   &   \\
            6     & 1 & 6 & 15 & 19 & 15 & 6  & 1  &    &   &   \\
            7     & 1 & 7 & 21 & 31 & 31 & 21 & 7  & 1  &   &   \\
            8     & 1 & 8 & 28 & 46 & 53 & 46 & 28 & 8  & 1 &   \\
            9     & 1 & 9 & 36 & 64 & 81 & 81 & 64 & 36 & 9 & 1
        \end{tabular}
    \end{center}
    \caption{Rascal triangle R2. Sequence \href{https://oeis.org/A374378}{\texttt{A374378}} in OEIS~\cite{sloane2003oeis}.}
    \label{tab:r2-triangle}
\end{table}
}
\end{examp}
\begin{examp}
    \emph{
        Rascal triangle R3 generated by $\rascalNumber{n}{k}{3}$
        \begin{table}[H]
    \begin{center}
        \begin{tabular}{c|cccccccccc}
            $n/k$ & 0 & 1 & 2  & 3  & 4   & 5   & 6  & 7  & 8 & 9 \\
            \hline
            0     & 1 &   &    &    &     &     &    &    &   &   \\
            1     & 1 & 1 &    &    &     &     &    &    &   &   \\
            2     & 1 & 2 & 1  &    &     &     &    &    &   &   \\
            3     & 1 & 3 & 3  & 1  &     &     &    &    &   &   \\
            4     & 1 & 4 & 6  & 4  & 1   &     &    &    &   &   \\
            5     & 1 & 5 & 10 & 10 & 5   & 1   &    &    &   &   \\
            6     & 1 & 6 & 15 & 20 & 15  & 6   & 1  &    &   &   \\
            7     & 1 & 7 & 21 & 35 & 35  & 21  & 7  & 1  &   &   \\
            8     & 1 & 8 & 28 & 56 & 69  & 56  & 28 & 8  & 1 &   \\
            9     & 1 & 9 & 36 & 84 & 121 & 121 & 84 & 36 & 9 & 1
        \end{tabular}
    \end{center}
    \caption{Rascal triangle R3. Sequence \href{https://oeis.org/A374452}{\texttt{A374452}} in OEIS~\cite{sloane2003oeis}.}
    \label{tab:r3-triangle}
\end{table}
}
\end{examp}


Since then, a lot of work has been done over the topic of rascal triangles.
Numerous identities and relations have been revealed.
For instance, a few combinatorial interpretations of rascal numbers provided at~\cite{gibbs2024two}, in particular,
these interpretations establish a relation between rascal numbers and combinatorics of binary words.
Several generalization approaches were proposed, namely generalized
and iterated rascal triangles~\cite{hotchkiss2019generalized,gregory2023iterated_Aequationes}.
In particular, the concept of iterated rascal numbers establishes a close connection between rascal numbers and binomial
coefficients.
