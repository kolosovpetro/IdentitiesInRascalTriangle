\documentclass[12pt,letterpaper,oneside,reqno]{amsart}
\usepackage{amsfonts}
\usepackage{amsmath}
\usepackage{amssymb}
\usepackage{amsthm}
\usepackage{float}
\usepackage{mathrsfs}
\usepackage{colonequals}
\usepackage[font=small,labelfont=bf]{caption}
\usepackage[left=1in,right=1in,bottom=1in,top=1in]{geometry}
\usepackage[pdfpagelabels,hyperindex,colorlinks=true,linkcolor=blue,urlcolor=magenta,citecolor=green]{hyperref}
\usepackage{graphicx}
\linespread{1.7}
\emergencystretch=1em
\usepackage{array}
\usepackage{etoolbox}
\apptocmd{\sloppy}{\hbadness 10000\relax}{}{}
\raggedbottom

\newcommand \anglePower [2]{\langle #1 \rangle \sp{#2}}
\newcommand \bernoulli [2][B] {{#1}\sb{#2}}
\newcommand \curvePower [2]{\{#1\}\sp{#2}}
\newcommand \coeffA [3][A] {{\mathbf{#1}} \sb{#2,#3}}
\newcommand \polynomialP [4][P]{{\mathbf{#1}}\sp{#2} \sb{#3}(#4)}

% ordinary derivatives
\newcommand \derivative [2] {\frac{d}{d #2} #1}                              % 1 - function; 2 - variable;
\newcommand \pderivative [2] {\frac{\partial #1}{\partial #2}}               % 1 - function; 2 - variable;
\newcommand \qderivative [1] {D_{q} #1}                                      % 1 - function
\newcommand \nqderivative [1] {D_{n,q} #1}                                   % 1 - function
\newcommand \qpowerDerivative [1] {\mathcal{D}_q #1}                         % 1 - function;
\newcommand \finiteDifference [1] {\Delta #1}                                % 1 - function;
\newcommand \pTsDerivative [2] {\frac{\partial #1}{\Delta #2}}               % 1 - function; 2 - variable;

% high order derivatives
\newcommand \derivativeHO [3] {\frac{d^{#3}}{d {#2}^{#3}} #1}                % 1 - function; 2 - variable; 3 - order
\newcommand \pderivativeHO [3]{\frac{\partial^{#3}}{\partial {#2}^{#3}} #1}
\newcommand \qderivativeHO [2] {D_{q}^{#2} #1}                               % 1 - function; 2 - order
\newcommand \qpowerDerivativeHO [2] {\mathcal{D}_{q}^{#2} #1}                % 1 - function; 2 - order
\newcommand \finiteDifferenceHO [2] {\Delta^{#2} #1}                         % 1 - function; 2 - order
\newcommand \pTsDerivativeHO [3] {\frac{\partial^{#3}}{\Delta {#2}^{#3}} #1} % 1 - function; 2 - variable;

% central factorials and related symbols
\newcommand \centralFactorial [2] {#1^{[#2]}}
\newcommand \fallingFactorial [2] {\left(#1 \right)^{\underline{#2}}}
\newcommand{\stirlingii}{\genfrac{\{}{\}}{0pt}{}}
\newcommand{\eulerianNumber}{\genfrac{\langle}{\rangle}{0pt}{}}

% for llceil coeffcient
\newcommand{\nobarfrac}{\genfrac{}{}{0pt}{}}
\def\llceil{\left\lceil\kern-3.5pt\left\lceil}
\def\rrfloor{\right\rfloor\kern-3.5pt\right\rfloor}
\newcommand \llceilCoefficient [3] {\llceil \nobarfrac{#1}{#2} \rrfloor_{#3}}

% rascal numbers etc
\newcommand \rascalNumber [3] {\binom{#1}{#2}_{#3}}
\newcommand \north[0] {\mathbf{North}}
\newcommand \south[0] {\mathbf{South}}
\newcommand \west[0] {\mathbf{West}}
\newcommand \east[0] {\mathbf{East}}

% 1-q pascal notation

\newcommand{\genstirlingI}[3]{%
    \genfrac{[}{]}{0pt}{#1}{#2}{#3}%
}
\newcommand{\genstirlingII}[3]{%
    \genfrac{\{}{\}}{0pt}{#1}{#2}{#3}%
}
\newcommand{\oneQBinomial}[3]{\genstirlingI{}{#1}{#2}^{#3}}

% free foot note
\let\svthefootnote\thefootnote
\newcommand\freefootnote[1]{%
    \let\thefootnote\relax%
    \footnotetext{#1}%
    \let\thefootnote\svthefootnote%
}


\newtheorem{thm}{Theorem}[section]
\newtheorem{cor}[thm]{Corollary}
\newtheorem{lem}[thm]{Lemma}
\newtheorem{examp}[thm]{Example}
\newtheorem{conj}[thm]{Conjecture}
\newtheorem{definition}[thm]{Definition}
\newtheorem{proposition}[thm]{Proposition}

\numberwithin{equation}{section}

\title[Identities in Iterated Rascal Triangles]
{Identities in Iterated Rascal Triangles}
\author[Petro Kolosov]{Petro Kolosov}
\address{Software Developer, DevOps Engineer}
\email{kolosovp94@gmail.com}
\urladdr{https://kolosovpetro.github.io}
\keywords{
    Pascal's triangle,
    Rascal triangle,
    Binomial coefficients,
    Binomial identities,
    Binomial theorem,
    Generalized Rascal triangles,
    Iterated rascal triangles,
    Iterated rascal numbers
}
\subjclass[2010]{11B25, 11B99}
\date{\today}
\hypersetup{
    pdftitle={Identities in Iterated Rascal Triangles},
    pdfsubject={
        Pascal's triangle,
        Rascal triangle,
        Binomial coefficients,
        Binomial identities,
        Binomial theorem,
        Generalized Rascal triangles,
        Iterated rascal triangles,
        Iterated rascal numbers,
        Number triangle,
        Arithmetic sequence
    },
    pdfauthor={Petro Kolosov},
    pdfkeywords={
        Pascal's triangle,
        Rascal triangle,
        Binomial coefficients,
        Binomial identities,
        Binomial theorem,
        Generalized Rascal triangles,
        Iterated rascal triangles,
        Iterated rascal numbers,
        Number triangle,
        Arithmetic sequence
    }
}
\begin{document}
    \begin{abstract}
        In this manuscript, we introduce new binomial identities in iterated Rascal triangles,
uncovering a connection between Vandermonde convolution and iterated Rascal numbers.
Additionally, we present novel identities involving the finite differences of iterated Rascal numbers
and binomial coefficients.
The manuscript also offers a proof of the row sums conjecture for iterated Rascal triangles.
Furthermore, we establish and explore the relationship between iterated Rascal triangles
and $(1,q)$-binomial coefficients, highlighting connections to relevant OEIS sequences.
All results are supported by supplementary Mathematica programs for validation.

    \end{abstract}

    \maketitle

    \tableofcontents

    \freefootnote{Sources: \url{https://github.com/kolosovpetro/IdentitiesInRascalTriangle}}
%    \section{Definitions}\label{sec:definitions}
%    \begin{definition}
    Iterated rascal number~\cite{gregory2023iterated}
    \begin{align}
        \rascalNumber{n}{k}{i} = \sum_{m=0}^{i} \binom{n-k}{m} \binom{k}{m}
    \end{align}
\end{definition}

\begin{definition}
    $(1,q)$-Binomial coefficient~\cite{sloane2004pascal}
    \begin{equation}
        \oneQBinomial{n}{k}{q} =
        \begin{cases}
            q & \mathrm{if} \; k=0, n=0 \\
            1 & \mathrm{if} \; k=0 \\
            0 & \mathrm{if} \; k > n \\
            \oneQBinomial{n-1}{k}{q} + \oneQBinomial{n-1}{k-1}{q}
        \end{cases}\label{eq:qbinomial-definition}
    \end{equation}
\end{definition}



    \section{Introduction} \label{sec:introduction}
    Rascal triangle is Pascal-like numeric triangle developed in 2010 by three middle school students,
Alif Anggoro, Eddy Liu, and Angus Tulloch~\cite{anggoro2010rascal}.
During math classes they were challenged to provide the next row for the following number triangle
\[
    \begin{array}{cccccccc}
        &   &   &   & 1 &   &   &   \\
        &   &   & 1 &   & 1 &   &   \\
        &   & 1 &   & 2 &   & 1 &   \\
        & 1 &   & 3 &   & 3 &   & 1 \\
        & & & & \dots & &
    \end{array}
\]

The teacher anticipated that the next row would match Pascal's triangle, such as ``1 4 6 4 1'',
by applying the binomial coefficient recurrence rule $\south = \east + \west$.
However, Anggoro, Liu, and Tulloch proposed that the next row should be ``1 4 5 4 1''.
Instead of using Pascal's triangle rule $\south = \east + \west$, they derived this new row using
a relation they termed the diamond formula
\begin{align}
    \south = \frac{\east \cdot \west + 1}{\north}
    \label{eq:diamond-rule}
\end{align}
By applying the recurrence relation from equation~\eqref{eq:diamond-rule},
the students successfully generated an entirely new triangular sequence,
now referred to as the Rascal triangle.
\begin{table}[H]
    \begin{center}
        \begin{tabular}{c|cccccccccc}
            $n/k$ & 0 & 1 & 2  & 3  & 4  & 5  & 6  & 7  & 8 & 9 \\
            \hline
            0     & 1 &   &    &    &    &    &    &    &   &   \\
            1     & 1 & 1 &    &    &    &    &    &    &   &   \\
            2     & 1 & 2 & 1  &    &    &    &    &    &   &   \\
            3     & 1 & 3 & 3  & 1  &    &    &    &    &   &   \\
            4     & 1 & 4 & 5  & 4  & 1  &    &    &    &   &   \\
            5     & 1 & 5 & 7  & 7  & 5  & 1  &    &    &   &   \\
            6     & 1 & 6 & 9  & 10 & 9  & 6  & 1  &    &   &   \\
            7     & 1 & 7 & 11 & 13 & 13 & 11 & 7  & 1  &   &   \\
            8     & 1 & 8 & 13 & 16 & 17 & 16 & 13 & 8  & 1 &   \\
            9     & 1 & 9 & 15 & 19 & 21 & 21 & 19 & 15 & 9 & 1
        \end{tabular}
    \end{center}
    \caption{Rascal triangle. Sequence \href{https://oeis.org/A077028}{\texttt{A077028}} in OEIS~\cite{sloane2002rascal}.}
    \label{tab:rascal-triangle-i-1}
\end{table}

For example, the fourth row is ``1 4 5 4 1'' because $4 = \frac{1 \cdot 3 + 1}{1}$ and $5 = \frac{3 \cdot 3 + 1}{2}$.
Moreover, the Rascal triangle, as presented in table~\eqref{tab:rascal-triangle-i-1},
represents the first and foundational instance of a new family of Pascal-like triangles.
This family, known as \textit{iterated Rascal triangles}, was first introduced by J. Gregory in her
master's thesis~\cite{gregory2022iterated_Aequationes}.

We define the $k$-th element in the $n$-th row of an iterated Rascal triangle as $\rascalNumber{n}{k}{i}$,
where $i$ represents the number of iterations.
The integer sequence produced by $\rascalNumber{n}{k}{i}$ is referred to as an \textit{iterated Rascal triangle $Ri$},
and each $\rascalNumber{n}{k}{i}$ is termed an \textit{iterated Rascal number}.
Therefore, the Rascal triangle shown in table~\eqref{tab:rascal-triangle-i-1} corresponds
to the iterated Rascal triangle $R1$, generated by the formula $\rascalNumber{n}{k}{1} = k(n-k)+1$.
While the iterated Rascal number $\rascalNumber{n}{k}{i}$ is defined by the diamond rule~\eqref{eq:diamond-rule},
which differs from the standard binomial coefficient recurrence,
it still maintains a significant connection with the binomial coefficients $\binom{n}{k}$,
as demonstrated by
\begin{equation}
    \rascalNumber{n}{k}{i} = \sum_{m=0}^{i} \binom{n-k}{m} \binom{k}{m}
    \label{eq:iterated-rascal-number}
\end{equation}
For example, $\rascalNumber{7}{4}{3}=35$, $\rascalNumber{12}{7}{5}=792$, $\rascalNumber{11}{5}{5}=462$.
\begin{examp}
    \emph{
        Rascal triangle R2 generated by $\rascalNumber{n}{k}{2}$
        \begin{table}[H]
    \begin{center}
        \begin{tabular}{c|cccccccccc}
            $n/k$ & 0 & 1 & 2  & 3  & 4  & 5  & 6  & 7  & 8 & 9 \\
            \hline
            0     & 1 &   &    &    &    &    &    &    &   &   \\
            1     & 1 & 1 &    &    &    &    &    &    &   &   \\
            2     & 1 & 2 & 1  &    &    &    &    &    &   &   \\
            3     & 1 & 3 & 3  & 1  &    &    &    &    &   &   \\
            4     & 1 & 4 & 6  & 4  & 1  &    &    &    &   &   \\
            5     & 1 & 5 & 10 & 10 & 5  & 1  &    &    &   &   \\
            6     & 1 & 6 & 15 & 19 & 15 & 6  & 1  &    &   &   \\
            7     & 1 & 7 & 21 & 31 & 31 & 21 & 7  & 1  &   &   \\
            8     & 1 & 8 & 28 & 46 & 53 & 46 & 28 & 8  & 1 &   \\
            9     & 1 & 9 & 36 & 64 & 81 & 81 & 64 & 36 & 9 & 1
        \end{tabular}
    \end{center}
    \caption{Rascal triangle R2. Sequence \href{https://oeis.org/A374378}{\texttt{A374378}} in OEIS~\cite{sloane2003oeis}.}
    \label{tab:r2-triangle}
\end{table}
}
\end{examp}
\begin{examp}
    \emph{
        Rascal triangle R3 generated by $\rascalNumber{n}{k}{3}$
        \begin{table}[H]
    \begin{center}
        \begin{tabular}{c|cccccccccc}
            $n/k$ & 0 & 1 & 2  & 3  & 4   & 5   & 6  & 7  & 8 & 9 \\
            \hline
            0     & 1 &   &    &    &     &     &    &    &   &   \\
            1     & 1 & 1 &    &    &     &     &    &    &   &   \\
            2     & 1 & 2 & 1  &    &     &     &    &    &   &   \\
            3     & 1 & 3 & 3  & 1  &     &     &    &    &   &   \\
            4     & 1 & 4 & 6  & 4  & 1   &     &    &    &   &   \\
            5     & 1 & 5 & 10 & 10 & 5   & 1   &    &    &   &   \\
            6     & 1 & 6 & 15 & 20 & 15  & 6   & 1  &    &   &   \\
            7     & 1 & 7 & 21 & 35 & 35  & 21  & 7  & 1  &   &   \\
            8     & 1 & 8 & 28 & 56 & 69  & 56  & 28 & 8  & 1 &   \\
            9     & 1 & 9 & 36 & 84 & 121 & 121 & 84 & 36 & 9 & 1
        \end{tabular}
    \end{center}
    \caption{Rascal triangle R3. Sequence \href{https://oeis.org/A374452}{\texttt{A374452}} in OEIS~\cite{sloane2003oeis}.}
    \label{tab:r3-triangle}
\end{table}
}
\end{examp}


Since then, a lot of work has been done over the topic of rascal triangles.
Numerous identities and relations have been revealed.
For instance, a few combinatorial interpretations of rascal numbers provided at~\cite{gibbs2024two}, in particular,
these interpretations establish a relation between rascal numbers and combinatorics of binary words.
Several generalization approaches were proposed, namely generalized
and iterated rascal triangles~\cite{hotchkiss2019generalized,gregory2023iterated}.
In particular, the concept of iterated rascal numbers establishes a close connection between rascal numbers and binomial
coefficients.



    \section{Conclusions}\label{sec:conclusions}
    Conclusions of your manuscript.

    \bibliographystyle{unsrt}
    \bibliography{IdentitiesInRascalTriangle}
    \noindent \textbf{Version:} \texttt{Local-0.1.0}

\end{document}
